%%%%%%%%%%%%%%%%%%%%%%%%%%%%%%%%%%%%%%%%%%%%%%%%%%%%%%%%%%
\documentclass[12pt,a4paper]{article}
%%%%%%%%%%%%%%%%%%%%%%%%%%%%%%%%%%%%%%%%%%%%%%%%%%%%%%%%%%
\usepackage{amssymb,geometry,graphicx,setspace, scalefnt, verbatim, amsfonts,bm,float,booktabs,multirow,amsmath,mathtools,color,dsfont,arydshln,subcaption}
\usepackage[left]{lineno}
\usepackage[round,sort]{natbib}
%\usepackage[nolists,tablesfirst]{endfloat}
\usepackage[hidelinks]{hyperref}
\usepackage{enumerate}
\usepackage{dsfont}
\usepackage{cancel}
\usepackage{ulem}
\geometry{a4paper,nohead,left=2.0cm,right=2.0cm,bottom=2.5cm,top=2.5cm}
\usepackage[hidelinks]{hyperref}
\usepackage{enumerate}
\usepackage{dsfont}
\usepackage{cancel}
\usepackage{ulem}
\geometry{a4paper,nohead,left=2.0cm,right=2.0cm,bottom=2.5cm,top=2.5cm}


\usepackage[ruled,vlined,linesnumbered]{algorithm2e}

% Defining useful commands for notation

\newcommand{\indep}{\perp \!\!\! \perp}
\newcommand{\dd}{\mathop{}\!\mathrm{d}}
\allowdisplaybreaks

\begin{document}
\onehalfspacing
\linenumbers


\begin{center}
    \textbf{Research visit}\\\smallskip

    Andr\'{e} F. B. Menezes\\
    Hamilton Institute, Maynooth University \\
    Last compiled: \today\vspace{-\medskipamount}

    %\textbf{Abstract}\vspace{0.1cm}
\end{center}


\section{Research plan}

The proposed research visit will be undertaken with Professor Chris Sherlock
from the department of Mathematics and Statistics at Lancaster University, UK.
The visit is scheduled to start on 12 January 2026 and will last for one month.

The goal of the visit is to work on sequential Monte Carlo (SMC) algorithms to
improve filtering and inference on high (or even moderate) dimension Markov
jump processes (MJP).

\section{Timeline}
A general work timeline is defined below by weeks.

I am not so sure on the timeline, as one month goes to fast.

When you mentioned to use particle MCMC scheme for inference, i have a couple
of questions, which is not clear:
1. The filtering algorithm will be construced using the SMC algorithm you wrote?


I am asking this, because it is not 100\% clear to me if particle MCMC
stands for using MCMC to perform inference on the parameters of the state-space
model, or use a MCMC move inside the particle filtering?

2. From the particle algorithm in 1., we used its estimate of the marginal
likelihood in the MCMC algorithm?
3. The MCMC algorithm is the random-walk Metropolis-Hastings?


\begin{itemize}
	\item[Week 1 (12th-16th January)]: coding a particle MCMC scheme for
	inference on a moderate-dimensional SDE.

	\item[Week 2 (19th-23th January)]:

	\item[Week 3 (26th-30th January)]: replicate the particle MCMC scheme for MJP
	simulated using tau-leap scheme.

	\item[Week 4 (2nd-6th February)]:

	\item[Week 5 (9th-13th February)]: increase the dimension of the MJP
\end{itemize}



\bibliographystyle{elsarticle-harv}
\bibliography{references}


\end{document}
